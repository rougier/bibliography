\documentclass[a4paper]{article}
\usepackage{fullpage}

%% Choose one of the following (if not choosing the default, 
%% viz., Computer Modern, font family):
%\usepackage{lmodern}
%\usepackage{mathpazo}
%\usepackage{kpfonts}
%\usepackage{mathptmx}
%\usepackage{times,mtpro2}
%\usepackage{stix}
%\usepackage{txfonts}
%\usepackage{newtxtext,newtxmath}
\usepackage{libertine} \usepackage[libertine]{newtxmath}

\usepackage[natbib=true,
            style=numeric,
            sorting=ydnt,
            backend=biber,
            bibencoding=utf8,
            giveninits=true,
            url=false, doi=false,
            defernumbers,
            maxcitenames=3,
            maxbibnames=100]{biblatex}

\addbibresource{rougier.bib}


\usepackage{tocloft}
\usepackage{etoc}       
\setcounter{secnumdepth}{4}

\usepackage{xcolor}
\definecolor{blendedblue}{rgb}{0.2, 0.2, 0.6}
\definecolor{blendedred}{rgb}{0.8, 0.2, 0.2}
\definecolor{lightgray}{rgb}{0.7, 0.7, 0.7}
\usepackage[bookmarks=true,
            breaklinks=true,
            pdfborder={0 0 0},
            citecolor=blendedblue,
            colorlinks=true,       % false: boxed links; true: colored links
            linkcolor=blendedblue,
            urlcolor=blendedblue,
            citecolor=blendedblue,
            linktocpage=false,
            hyperindex=true,
            linkbordercolor=white]{hyperref}
\usepackage{hyperref}
\hypersetup{colorlinks=true}


% \usepackage[rmargin=4.5cm,lmargin=2cm,bmargin=2.5cm,
%             marginparwidth=3cm,twoside=false]{geometry}


% ------------------------------------------------------------------------------------
% Sidenotes
% ------------------------------------------------------------------------------------
\DeclareDatamodelFields[type=field, datatype=literal, skipout=false]{sidenote}
\DeclareFieldFormat{sidenote}{\marginpar{\footnotesize \textsc{\textcolor{lightgray}{#1}}}}

\DeclareBibliographyCategory{Highlight}

\usepackage{tikz} % Tikz basic
\usetikzlibrary{tikzmark} % Gives \tikzmark for markers
\usetikzlibrary{calc} % Node position calculations
\definecolor{HighlightForeground}{HTML}{000000}
\definecolor{HighlightBackground}{HTML}{F5F5F5}
\def\bibitemstyle{\color{HighlightForeground}}

% New \highlight command for bibitem highlighting
% Takes parameter {entrykey} to make \tikzmark pairs unique
% \highlight command should appear before the text to highlight so it is set behind
\newcommand{\highlight}[1]{%
\begin{tikzpicture}[remember picture, overlay]
\coordinate (begin) at ($ (pic cs:#1beg) + (-.5ex,2ex) $);
\coordinate (end) at ($ (pic cs:#1end) + (0,-.8ex) $);
\coordinate (penult) at ($ (begin |-  end) + (0,-.5ex) $);
\coordinate (right) at ($ (begin) + (\linewidth+1ex,0) $);
\fill[HighlightBackground] (begin) -- (right |- begin)
                                   -- (right |- end)
                                   -- ( begin |- end)
                                   -- cycle;
\end{tikzpicture}}


\AtEveryBibitem{
  \ifcategory{Highlight}%
    {\highlight{\thefield{entrykey}}\tikzmark{\thefield{entrykey}beg}\bibitemstyle}{}
  \printfield{sidenote}%
}
\renewbibmacro*{finentry}{\finentry\tikzmark{\thefield{entrykey}end}}

% ------------------------------------------------------------------------------------
% No square brackets
% ------------------------------------------------------------------------------------
\DeclareFieldFormat{labelnumberwidth}{#1}


% ------------------------------------------------------------------------------------
% From TeX stack exchange
% "Is it possible to make a reference clickable without showing its DOI or URL field?"
% ------------------------------------------------------------------------------------
\newcommand{\doiorurl}{%
  \iffieldundef{doi}
    {\iffieldundef{url}
       {}
       {\strfield{url}}}
    {http://dx.doi.org/\strfield{doi}}%
}
\newcommand{\myhref}[1]{%
 \ifboolexpr{%
   test {\ifhyperref}
   and
   not test {\iftoggle{bbx:url}}
   and
   not test {\iftoggle{bbx:doi}}
  }
  {\href{\doiorurl}{#1}}
  {#1}%
}
\DeclareFieldFormat{title}{\myhref{\mkbibemph{#1}}}
\DeclareFieldFormat
  [article,inbook,incollection,inproceedings,patent,thesis,unpublished]
  {title}{\myhref{\mkbibquote{#1\isdot}}}


% ------------------------------------------------------------------------------------  

% ------------------------------------------------------------------------------------
\begin{document}

\nocite{*}

\section*{Nicolas P. Rougier's bibliography}
\vspace{-3mm}
Latest update on \today.

\addtocategory{Highlight}{
  Science:2018,
  rougier:book:2018,
  rougier:2018a,
  rougier:2018b,
  rougier:2018c,
  benureau:2018,
  strock:2018,
  rougier:invited:2018a,
  rougier:invited:2018b,
  alexandre:chapter:2018,
}

\defbibheading{subbibliography}[\refname]{\subsubsection*{#1}}

\defbibfilter{book}{keyword={book}}
\defbibfilter{preprint}{keyword={preprint}}
\defbibfilter{journal}{keyword={journal}}
\defbibfilter{correspondence}{keyword={correspondence}}
\defbibfilter{chapter}{type=incollection}
\defbibfilter{conference}{type=inproceedings}
\defbibfilter{tutorial}{keyword={tutorial}}
\defbibfilter{popular}{keyword={popular}}
\defbibfilter{invited}{keyword={invited}}
\defbibfilter{external}{keyword={external}}
\defbibfilter{replication}{keyword={replication}}
\defbibfilter{lecture}{keyword={lecture}}


%\noindent \textbf{Categories}
\etocsetnexttocdepth{6}
\etocsettocstyle{\subsubsection*{}}{}
\cftsubsubsecindent 0pt
\localtableofcontents

\subsubsection*{Preprints}

\addcontentsline{toc}{subsubsection}{Preprints}
\begin{refcontext}[labelprefix=PP.]
% \printbibliography[heading=subbibliography, title={Preprints},  filter=preprint]
\printbibliography[heading=none, filter=preprint]
\end{refcontext}

\subsubsection*{Replications} % \small \normalfont (peer reviewed)}
\addcontentsline{toc}{subsubsection}{Replications}
\begin{refcontext}[labelprefix=RE.]
\printbibliography[heading=none, filter=replication]
\end{refcontext}

\subsubsection*{Correspondences} %  \small \normalfont (peer reviewed)}
\addcontentsline{toc}{subsubsection}{Correspondences}
\begin{refcontext}[labelprefix=CO.]
\printbibliography[heading=none, filter=correspondence]
\end{refcontext}


\subsubsection*{Highlights}
\addcontentsline{toc}{subsubsection}{Highlights}
\begin{refcontext}[labelprefix=HI.]
\printbibliography[heading=none, filter=external]
\end{refcontext}


\subsubsection*{Journals}
\addcontentsline{toc}{subsubsection}{Journals}
\begin{refcontext}[labelprefix=IJ.]
\printbibliography[heading=none, filter=journal]
\end{refcontext}


\subsubsection*{Book \& Thesis}
\addcontentsline{toc}{subsubsection}{Book \& Thesis}
\begin{refcontext}[labelprefix=BK.]
\printbibliography[heading=none, filter=book]
\end{refcontext}

\subsubsection*{Book Chapters}
\addcontentsline{toc}{subsubsection}{Book Chapters}
\begin{refcontext}[labelprefix=CH.]
\printbibliography[heading=none, filter=chapter]
\end{refcontext}

\subsubsection*{Conferences/Short Papers/Posters}
\addcontentsline{toc}{subsubsection}{Conferences/Short Papers/Posters}
\begin{refcontext}[labelprefix=CI.]
\printbibliography[heading=none, filter=conference]
\end{refcontext}


\subsubsection*{Science Outreach}
\addcontentsline{toc}{subsubsection}{Science Outreach}
\begin{refcontext}[labelprefix=SO.]
\printbibliography[heading=none, filter=popular]
\end{refcontext}


\subsubsection*{Invited Talks}
\addcontentsline{toc}{subsubsection}{Invited Talks}
\begin{refcontext}[labelprefix=IV.]
\printbibliography[heading=none, filter=invited]
\end{refcontext}

\subsubsection*{Invited Lectures}
\addcontentsline{toc}{subsubsection}{Invited Lectures}
\begin{refcontext}[labelprefix=IL.]
\printbibliography[heading=none, filter=lecture]
\end{refcontext}


\subsubsection*{Online Tutorials}
\addcontentsline{toc}{subsubsection}{Online Tutorials}
\begin{refcontext}[labelprefix=TU.]
\printbibliography[heading=none, filter=tutorial]
\end{refcontext}


\end{document}



